% PLANTILLA APA7
% Creado por: Isaac Palma Medina
% Última actualización: 25/07/2021
% @COPYLEFT

% Fuentes consultadas (todos los derechos reservados):  
% Normas APA. (2019). Guía Normas APA. https://normas-apa.org/wp-content/uploads/Guia-Normas-APA-7ma-edicion.pdf
% Tecnológico de Costa Rica [Richmond]. (2020, 16 abril). LaTeX desde cero con Overleaf (1 de 3) [Vídeo]. YouTube. https://www.youtube.com/watch?v=kM1KvHVuaTY Weiss, D. (2021). 
% Formatting documents in APA style (7th Edition) with the apa7 LATEX class. https://ctan.math.washington.edu/tex-archive/macros/latex/contrib/apa7/apa7.pdf @COPYLEFT

%+-+-+-+-++-+-+-+-+-+-+-+-+-++-+-+-+-+-+-+-+-+-+-+-+-+-+-+-+-+-++-+-+-+-+-+-+-+-+-+

% Preámbulo
\documentclass[stu, 12pt, letterpaper, donotrepeattitle, floatsintext, natbib]{apa7}
\usepackage[utf8]{inputenc}
\usepackage{comment}
\usepackage{marvosym}
\usepackage{graphicx}
\usepackage{float}
\usepackage[normalem]{ulem}
\usepackage[spanish]{babel} 
\selectlanguage{spanish}
\useunder{\uline}{\ul}{}
\newcommand{\myparagraph}[1]{\paragraph{#1}\mbox{}\\}

% Portada
\thispagestyle{empty}
\title{\Large Evaluación}
\author{Rodríguez García Eduardo Alberto} % (autores separados, consultar al docente)
\authorsaffiliations{Universidad Autónoma de Guadalajara}
\course{Reconocimiento de patrones}
\professor{Dr. Carlos Alejandro de Luna Ortega}
\duedate{\today}
\begin{document}
    \maketitle


    % Índices
    \pagenumbering{roman}
    % Contenido
    \renewcommand\contentsname{\largeÍndice}
    \tableofcontents
    \setcounter{tocdepth}{2}
    \clearpage
    % Figuras
    \renewcommand{\listfigurename}{\largeÍndice de fíguras}
    \listoffigures
    \clearpage
    % Tablas
    \renewcommand{\listtablename}{\largeÍndice de tablas}
    \listoftables
    \clearpage

    % Cuerpo
    \pagenumbering{arabic}

    \section{\large Definir los siguientes conceptos}

        \subsection{Patrón}
            Sirve de muestra para categorizar alguna cosa como similar, esto puede ser por sus características generales, objetos recurrentes, sucesos, entre otros.

        \subsection{Reconocimiento de patrones}
            Es el procesamiento de información que da solución a un amplio rango de problemas. Es una disciplina científica que tiene como objetivo el clasificar objetos en un número específico de categorías o clases. Dependiendo de la aplicación, estos objetos pueden ser imágenes, sonidos, o señales que pueden ser clasificadas.

        \subsection{Características}
            Conjunto de cualidades que poseen la misma clase de elementos.


        \subsection{Vectores de Características}
            Conjunto de propiedades que distinguen los objetos de las clases.

        \subsection{Clasificador}
            Es una función que mapea el valor de las características en un conjunto de categorías o clases.

        \subsection{Aprendizaje}
            Parámetros ajustables que usa el clasificador.

    \section{Defina dos ejemplos donde se utilizaría el reconocimiento de patrones}
        \subsection{Visión de máquina}
            Tiene que ver con la captura de imágenes con ayuda de cámaras digitales y la interpretación automática de lo que está en la imagen.

        \subsection{Reconocimiento de caracteres}
            Esta aplicación está relacionada con la transformación de textos impresos o manuscritos a formato digital, lo que permite una mayor flexibilidad en la manipulación de la información.

        \subsection{Reconocimiento de voz}
            La aplicación en este caso tiene que ver con la construcción de máquinas que puedan reconocer la información hablada.

    \section{Explique la diferencia entre reconocimiento de patrones, data science, y minería de datos}
        El data science se encarga de estudiar información para poder convertirla en un recurso valioso en la creación de negocios y estrategias.

        El data mining es una técnica que se usa en data science para extraer datos.

        El reconcimiento de patrones es una técnica que se usa en data science para clasificar.

    \section{Defina que es el aprendizaje supervisado y el aprendizaje no supervisado}
        \subsection{Aprendizaje supervisado}
            Consiste en clasificar nuevos objetos basándose en la información de una muestra ya clasificada.

        \subsection{Aprendizaje no supervisado}
            Consiste en dada una muestra no clasificada encontrar la clasificación de la misma.

    \section{Defina dos ejemplos de aprendizaje supervisado y no supervisado}
        \begin{enumerate}
            \item Identificación de rostros.
            \item Búsqueda de petróleo.
            \item Predicción de magnitudes máximas de terremotos.
        \end{enumerate}
    

\end{document}